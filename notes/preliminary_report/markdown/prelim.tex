\section{Meta-Description}

The preliminary report is effectively a summary of progress at the end
of Teaching Period 1, and a plan of the intended work during Teaching
Period 2.

The preliminary report should not exceed 15 pages (excluding the log
book), and MUST contain the following:

\begin{itemize}
\itemsep1pt\parskip0pt\parsep0pt
\item
  a brief summary (1 page maximum) of the achievements to date.
\item
  a description of the project, including the main aims and/or
  objectives.
\item
  an introduction, putting the project into context with other work in
  the same subject area, incorporating a comprehensive review of
  relevant literature, including references to books, journal articles,
  conference proceedings, manuals, and the Internet.
\item
  Evidence of use of the library and online databases should be
  included.
\item
  an individual project plan for each student for the intended work
  during Teaching Period 2, including specific aims, goals and
  timescales, and a clear division of labour for students working
  together on a joint project. A Gantt chart or similar visualization
  aid should be included.
\item
  a detailed discussion of any ethical issues (environmental, societal,
  legal, professional or medical) that may be applicable to the proposed
  research, and any relevant industrial standards and Health and Safety
  requirements.
\end{itemize}

Colin's Tips:

\begin{itemize}
\itemsep1pt\parskip0pt\parsep0pt
\item
  Focus on what you've learnt so far
\item
  Provide a detailed breakdown of how you expect to proceed in TP2, this
  will be compared with the final report
\item
  2nd examiner may not be familiar with comms
\end{itemize}

\section{Summary of acheivements to date}

\begin{itemize}
\itemsep1pt\parskip0pt\parsep0pt
\item
  The author has familiarised himself with relevent communications
  theory, including modulation schemes, probability theory and optimum
  detector design.
\item
  The author has become acquainted with the Mathematica programming
  environment and used it to program a number of mathematical
  simulations describing communications problems.
\item
  The author has examined the effects of receiver timing error on BPSK
  and 4-PAM modulation schemes using root raised cosine filters, and
  demonstrated sub-optimal detection performance with 4-PAM using
  standard detector designs.
\item
  The author has characterised the change of optimum decision region
  boundaries in the presence of non-deterministic timing errors
  following a known probablistic model.
\end{itemize}

\section{Introduction and background to the project}

This project seeks to examine the effects of receiver timing error in
radio communications systems. Understanding how radio system performance
decreases in sub-optimal conditions is key to assessing a system's
performance and designing more robust and performng radio systems. The
author hopes to be able to characterise the effects of timing error such
that a receiver may be optimised to account for it.

A typical radio communications system consists of a transmitter, a
receiver and a communications channel, that may contain any number of
non-idealities. For this reason, if a transmitter sends a particular
symbol down a real communications channel, the received symbol may be
substantially different to the sent symbol. Therefore, careful attention
is required to the design of the receiver symbol detector.

If we imagine a binary transmission system that sends one of two
possible signals: a 1V$_{RMS}$ wave if a `0' is to be sent, and a
3V$_{RMS}$ if a `1' is to be sent. After being distorted by the
communications channel, the receiver could in theory see a signal of any
amplitude, an must make a decision as to what amplitude was originally
sent. It would be helpful to know the probability density function (PDF)
of the received signal, ie. the probability of receiving a signal
amplitude if a known amplitude was sent. If we assume the communications
channel distorted the signal by adding zero-mean white Gaussian noise,
the received signal PDF will be a Gaussian distribution centred on the
sent signal amplitude.

\begin{figure}[htbp]
\centering
\includegraphics{4-PAM_PDF.png}
\caption{Gaussian noise corrupts a sent signal, resulting in a
probability density function for each possible sent symbol}
\end{figure}

\begin{figure}[htbp]
\centering
\includegraphics{4-PAM_samples.png}
\caption{As an example, 2000 of each symbol has been send down the
communications channel. The received values distorted by noise are
plotted above. Note the overlap in received values corresponding to both
symbols: it is impossible to detect the signal with 100\% accuracy.}
\end{figure}

An ideal detector would always pick whichever signal was most likely to
be sent. Therefore, the treshold between picking one value or the other
will be where both symbols are equally likely to be sent, the point of
intersection of both PDF's. Since the Gaussian distribution is symmetric
about its mean, in this case the treshold (or \textbf{Decision Region
Boundary}) is exactly midway between both sent amplitudes. Intuitively
this makes sense: it says that if we receive a signal amplitude, we
should assume whichever possible sent amplitude is nearest. However this
is not generally the case, and in more complicated cases picking the
point of intersection of both PDF's is a good solution.

One issue that complicates detection is \textbf{inter-symbol
interference} (ISI). It is possible for signals representing symbols in
the future or past to bleed into the current symbol clock period,
distorting the signal further. For this reason transmtters and receivers
are designed such that they apply a \textbf{raised cosine function} that
is fully transparent at exactly the symbol's sample time, and blocks
completely at every other symbol's sample time, ie at every interval of
$T_{clk}$ for $T \neq 0$. If both receiver and transmitter are perfectly
synchronised, this ensures that the receiver will only see signals
corresponding to the current transmitted symbol, after distortion via
the communications channel.

\begin{figure}[htbp]
\centering
\includegraphics{rrc_sync.png}
\caption{A filter with a root raised cosine function is often used, as
it evaluates to 1 at the current sample time and 0 at all other sample
times}
\end{figure}

\begin{figure}[htbp]
\centering
\includegraphics{rrc_err.png}
\caption{If a timing offset is added to the root raised cosine it no
longer evaluates to 0 or 1 at the sampling points. This results in
reduced receiver performance when the receiver and transmitter are not
properly synchronised.}
\end{figure}

If the receiver and the transmitter are poorly synchronised, however,
the root cosine filter response is shifted in time with respect to the
receiver and looses some of its essential property. Current literature
has considered\ldots{} However, the effects of a poorly synchronised
receiver on detection performance has not to date been publicised. This
project seeks to examine this matter further, and ascertain whether
design decisions can be made to mitigate its effects.

\section{Description of the project}

To get started on the project I began by reading up on the underlying
theory. I consulted Proakis \emph{(ref)} and learnt about probability
theory, coding theory and optimum receiver design. I became familiar
with the concept of representing received signal characteristics using
probablistic models, and the rationale for doing so in the Additive
White Gaussian Noise (AWGN) case described before. I learnt about
designing receivers using both Maximum Likelihood and Maximum A
Posteriori optimisation, and how this might be done using a known
probablistic model for the received signal. I also spent some time
understanding research done by a UCC graduate, David McCarthy, into
estimating these probablistic models using a Gram-Charlier probability
distribution. Additionally, I started learning the Mathematica language
from scratch, with emphasis on statistics and plot generation.

In order to familiarise myself further with these concepts, I built a
simple model for a two-symbol BPSK communications system in Mathematica.
I extended this to the 4-symbol 4-PAM system. With some help I also
implemented the Gram-Charlier distribution. Using these programs as a
base I then wrote a simulation that would a PDF of the receiver input
with a known timing offset. Using this, I was able to examine how the
sent symbol is distorted by the channel and timing error, and found that
increasing amounts of offset reduced the effective amplitude of the
signal while increasing noise due to ISI. Additionally, it was found
that the optimum decision region boundary was decreased by a factor
equal to the value of the root raised cosine function at the timing
offset.

My attention at this stage turned to increasing the speed of the
simulations, and much work was done studying Mathematica's potential for
parallisation. I was able to acquire remote access to a number of Unix
machines, and ported the simulations to run across multiple machines,
paying attention to taking advantage of the dual-core processors and
making the code robust to premature termination (due to power outages,
illiterate students etc.).

A more realistic model was created by assuming that the timing error is
not a fixed value, but could be described by a Tikhonov distribution.
Two approaches were taken: an initial, quick approach was to calculated
the optimum decision region boundary of each timing error, and average
this over the probability of each timing error occuring. This assumes
that multiple possible optimum decision region boundary locations could
be averaged to give the overall optimum decision region boundary
location, which I learnt was not generally the case, and this was later
proven analytically by David. I therefore tried a second, more realistic
approach, where for each simulated transmission a new timing offset was
chosen from the Tikhonov distribution. An optimum decision region
boundary was then determined for multiple Tikhonoc distribution widths
(or variances). It was found that the optimum decision region boundary
decreased as the variance of the Tikhonov distribution increased.

To summarise, my work in TP1 helped introduce me to radio communications
systems, and concepts of signal corruption and optimum detectors. I
learnt how to translate signal transmission into mathematical
simulations, and how to extract meaningful conclusions from the output.
Using the theory and simulations developped during this time I was able
to demonstrate a changing optimum decision region boundary in the
presence of a non-deterministic timing offset using 4-PAM signalling. It
is believed this behavior can be generalised to L-PAM signalling. This
will serve as a foundation from which to explore the effects of a
non-deterministic timing offse in more general cases.

\section{Plan for Teaching Period 2}

In TP2, I aim to extend the scope of my project to address the issue of
timing error offset in communications channels subject to fading effects
such as in urban areas or for very long-range communications. In either
case, line-of-sight communication is very weak or non-existant, and the
signal is scattered and received as a combination of multiple
``bounced'' signals with different propagation delays and amplitudes. I
will model these using a Rayleigh distribution, and asuming an Equal
Gain Combining (EGC or additive) receiver, determine the effects of
timing error described above on this type of model. I will stick to
L-PAM signalling, as PSK signalling formats rely solely on phase
detection and therefore the optimum decision region boundaries are
unaffected by amplitude changes.

Once we have a description of the optimum detector in the presence of
Rayleigh fading, the next goal will be to compare it to the optimum
detector described in (\emph{ref}), which assumes perfect
synchronisation (ie. no timing offset).

A project plan is presented here as a guide to how work is expected to
proceed in TP2. This plan is to be considered a mere guide, as
Hopfstadter's Rule ensures that no project plan can be accepted as
gospel truth. In order to account for this timescales have been made
purposefully conservative. Should the work laid out below be terminated
early, additional exploration of the effects of non-deterministic timing
error on other communications systems will be undertaken.

\begin{longtable}[c]{@{}llll@{}}
\hline\noalign{\medskip}
\begin{minipage}[b]{0.26\columnwidth}\raggedright
Description
\end{minipage} & \begin{minipage}[b]{0.15\columnwidth}\raggedright
Time Allowed
\end{minipage} & \begin{minipage}[b]{0.13\columnwidth}\raggedright
Start Date
\end{minipage} & \begin{minipage}[b]{0.46\columnwidth}\raggedright
Goals
\end{minipage}
\\\noalign{\medskip}
\hline\noalign{\medskip}
\begin{minipage}[t]{0.26\columnwidth}\raggedright
Review of Rayleigh fading
\end{minipage} & \begin{minipage}[t]{0.15\columnwidth}\raggedright
1 week
\end{minipage} & \begin{minipage}[t]{0.13\columnwidth}\raggedright
6 Jan
\end{minipage} & \begin{minipage}[t]{0.46\columnwidth}\raggedright
\begin{itemize}
\itemsep1pt\parskip0pt\parsep0pt
\item
  Develop an understanding of Rayleigh fading.
\item
  Have built a basic Mathematica model of Rayleigh fading assuming
  perfect synchronisation.
\end{itemize}
\end{minipage}
\\\noalign{\medskip}
\begin{minipage}[t]{0.26\columnwidth}\raggedright
Implement Rayleigh fading model with timing error
\end{minipage} & \begin{minipage}[t]{0.15\columnwidth}\raggedright
1 week
\end{minipage} & \begin{minipage}[t]{0.13\columnwidth}\raggedright
13 Jan
\end{minipage} & \begin{minipage}[t]{0.46\columnwidth}\raggedright
\begin{itemize}
\itemsep1pt\parskip0pt\parsep0pt
\item
  Extend previous model to account for timing errors following a
  Tikhonov distribution model.
\item
  Evaluate optimum decision region boundary in the presence of timing
  error offsets and Rayleigh fading.
\end{itemize}
\end{minipage}
\\\noalign{\medskip}
\begin{minipage}[t]{0.26\columnwidth}\raggedright
Additional simulation time
\end{minipage} & \begin{minipage}[t]{0.15\columnwidth}\raggedright
1 week
\end{minipage} & \begin{minipage}[t]{0.13\columnwidth}\raggedright
20 Jan
\end{minipage} & \begin{minipage}[t]{0.46\columnwidth}\raggedright
\begin{itemize}
\itemsep1pt\parskip0pt\parsep0pt
\item
  Characterise the effect of non-deterministic timing error on the
  optimum decision region boundary for L-PAM signalling in the presence
  of Rayleigh fading.
\end{itemize}
\end{minipage}
\\\noalign{\medskip}
\begin{minipage}[t]{0.26\columnwidth}\raggedright
Compare described receiver to optimum receiver described in literature
(\emph{ref})
\end{minipage} & \begin{minipage}[t]{0.15\columnwidth}\raggedright
2 weeks
\end{minipage} & \begin{minipage}[t]{0.13\columnwidth}\raggedright
27 Jan
\end{minipage} & \begin{minipage}[t]{0.46\columnwidth}\raggedright
\begin{itemize}
\itemsep1pt\parskip0pt\parsep0pt
\item
  Provide a detailed comparison of the optimum described above, to the
  optimum receiver described in (\emph{ref}), paying particular
  attention to performance in the presence of non-deterministic timing
  error.
\end{itemize}
\end{minipage}
\\\noalign{\medskip}
\hline
\end{longtable}

(\emph{insert Gantt chart here - pdfgantt})

\section{Discussion of ethical issues}

In communications systems, as in any area seeing considerable
technological developments, the question of whether these developments
are ethically sound naturally arises. The advent and spread of radio
throughout the 20th century brought about an era of increased
connectedness and information transfer, as information could be rapidly
spread across large distances with little effort. The advent of the
internet in 1969 (\emph{disputed - check this}) and the mobile phone
boom of the 90s accelerated these changes as almost-instant,
asynchronous and on-demand information transfer became available to the
general public, and current Web 2.0 developments such as social
networking are but continuations of this trend. With the smart-phone
industry putting internet access into the pockets of consumers, many
believe that these changes, driven by advances in communications
technology, have had a considerable impact on our societies. While the
positives are too numerous to note, the skeptic would also readily see
the disadvantages.

Engineering developments can lend themselves to applications with both
positive and negative intentions, as with any scientific advancement.
Improving wireless communications can play into the hands of military
and terrorist leaders, as cheap and reliable wireless communications can
aid the organisation and execution of manoeuvres. In addition the remote
detonation of explosives using mobile phones is considered a real threat
to many cities. (\emph{ref}) Increase in mobile cell phone usage has
also helped drug trafficking (\emph{check}) gangs, as they are able to
operate a decentralised operation with little face-to-face
communication. (\emph{ref})

Paradoxically, these same developments have created privacy concerns as
most citizens find themselves communicating through the internet and the
cell phone network. At the bottom of the internet heirarchy, endpoint
routers are naturally trusting, and prone to exploitation; in addition,
they are generally visible to all other internet users. At the top, most
communications are routed through a small number of Tier 1 networks. As
the distance over which an individual can send a piece of information to
its intended recipient has increased, so has the number of individuals
capable of intercepting this information. Thus as users embrace the
internet and smart phones, they also find themselves increasingly
exposed to spying from a number of groups with varying intentions. Even
outside of the ``online'' world, increasing the efficiency of radio
receivers can lead to the exploitation of security-sensitive short-range
communications. (\emph{eg. radio credit cards})

\begin{itemize}
\itemsep1pt\parskip0pt\parsep0pt
\item
  Two-sided coin: Use of increase range for red or blue team
\item
  Privacy issues
\item
  Liability issues involving over-reliance on RF comms
\end{itemize}

(\emph{See
http://www2.unescobkk.org/elib/publications/ethic\_in\_asia\_pacific/239\_325ETHICS.PDF})
